%-------------------------------------------------------------------------------
% HAP
%-------------------------------------------------------------------------------
\documentclass{ipol}
\ipolSetTitle{HAP: Hierarchical Affinity Propagation}
\ipolSetAuthors{Nelle Varoquaux}
\ipolSetYear{2012}
\ipolSetMonth{3}
\ipolSetDay{1}
\ipolSetID{gjmr-lsd}

\usepackage{hyperref,verbatim,graphicx,amsmath,amssymb,amssymb,dsfont}
\usepackage[ruled,linesnumbered]{algorithm2e}
\newtheorem{theorem}{Theorem}

\bibliographystyle{acm}

\begin{document}

%-------------------------------------------------------------------------------
\begin{abstract}

Clustering data by finding representative points is an important task of data
analysis. \cite{frey07affinitypropagation} introduces a novel algorithm based
on passing messages to find such points, called "exemplars". \cite{hap}
extended this algorithm to find hierarchical layers of exemplars. We present
this method, called Hierarchical Affinity Propagation (HAP).

\end{abstract}

%-------------------------------------------------------------------------------
\begin{ipolCode}
\end{ipolCode}

%-------------------------------------------------------------------------------
\begin{ipolSupp}
\end{ipolSupp}

%-------------------------------------------------------------------------------
\section{Introduction}
%-------------------------------------------------------------------------------

Finding exemplars is a critical step in pattern recognition. 

Standard clustering methods include randomly choosing a number of points from
the dataset, and iteratively refining clusters. When cluster centers are
chosen among datapoints, they are called "exemplars". Such methods, sensitive to the
initialisation, often fall in local minimums, and have to be rerunned several
times.

Affinity propagation takes a noval approach, and considers all datapoints as
potential exemplars.

\subsection{Preliminaries}

- preferences
- similarities

- responsabilities
- availabilities
- exemplars

\section{Affinity Propagation}

\begin{algorithm}[H]
     \SetLine

     initialization\;
     \caption{Affinity Propagation}
\end{algorithm}


\section{Hierarchical Affinity Propagation}

\bibliography{ipol_hap}

\end{document}
%-------------------------------------------------------------------------------
