%-------------------------------------------------------------------------------
% HAP
%-------------------------------------------------------------------------------
\documentclass{ipol}
\ipolSetTitle{HAP: Hierarchical Affinity Propagation}
\ipolSetAuthors{Nelle Varoquaux}
\ipolSetYear{2012}
\ipolSetMonth{3}
\ipolSetDay{1}
\ipolSetID{gjmr-lsd}

\usepackage{hyperref,verbatim,graphicx,amsmath,amssymb,amssymb,dsfont}
\usepackage[ruled,linesnumbered]{algorithm2e}
\newtheorem{theorem}{Theorem}

\bibliographystyle{acm}

\begin{document}

%-------------------------------------------------------------------------------
\begin{abstract}

Clustering data by finding representative points is an important task of data
analysis. \cite{frey07affinitypropagation} introduces a novel algorithm based
on passing messages to find such points, called "exemplars". \cite{hap}
extended this algorithm to find hierarchical layers of exemplars. We present
this method, called Hierarchical Affinity Propagation (HAP).

\end{abstract}

%-------------------------------------------------------------------------------
\begin{ipolCode}
\end{ipolCode}

%-------------------------------------------------------------------------------
\begin{ipolSupp}
\end{ipolSupp}

%-------------------------------------------------------------------------------
\section{Introduction}
%-------------------------------------------------------------------------------

Finding exemplars is a critical step in pattern recognition.

Standard clustering methods include randomly choosing a number of points from
the dataset, and iteratively refining clusters. When cluster centers are
chosen among datapoints, they are called "exemplars". Such methods, sensitive to the
initialisation, often fall in local minimums, and have to be rerunned several
times.

Affinity propagation (AP) takes a noval approach, and considers all datapoints
as potential exemplars. By passing messages reflecting the affinity between
datapoints, clusters emerge gradually during the procedure.

\subsection{Preliminaries}

Affinity propagation takes in input two elements: similarities and
preferences. The similarity $s_{ij}$ defines how well $k$ is suited to be
$i$'s exemplar, ie how $i$ and $j$ are similar, and the preference $p_i$ sets
how well $j$ is likely to be chosen as exemplar. Affinity propagation does not
require similarities to be metric, nor symmetric; $s_{ij}$ can be set to the
negative euclidean distance between two points.

Two notions are derived from the similarities and preferences:
responsabilities and availabilities. These two messages are computed
iteratively until convergence.

The responsability $r_{ik}$ of data point $i$ to candidate exemplar $j$ corresponds to
how well $k$ is suited to be an exemplar for $i$, compared to all the other
exemplar candidates. \\
\begin{equation*}
r_{ij} = s_{ij} - \max_{k \neq j} (a_{ik} + s_{ik})
\end{equation*}

The availability $a_{ik}$ of candidate exemplar $k$ to point $i$ shows how
appropriate $i$ to choose point $k$ as an exemplar taking in account all the
other points $j$ that would take $k$ as an exemplar. The self-availability
$a_{kk}$ reflects the probability that point $k$ is an exemplar based on the
positive responsability send from all points to $k$.

\begin{equation*}
a_{ij} = \begin{cases}
	    p_j + \sum_{k \neq j} \max(0, r_{kj}) &  i = j \\
	    \min ( 0, p_j + r_{jj} + \sum_{k \notin \{i, j\} } ) \max (0, r_{ij})
	    & i \neq j\\
	 \end{cases}
\end{equation*}


\section{Affinity Propagation}

\begin{algorithm}[H]
     \SetLine

     initialization\;
     \caption{Affinity Propagation}
\end{algorithm}


\section{Hierarchical Affinity Propagation}

\bibliography{ipol_hap}

\end{document}
%-------------------------------------------------------------------------------
